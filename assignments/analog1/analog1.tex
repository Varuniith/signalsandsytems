% \iffalse
\let\negmedspace\undefined
\let\negthickspace\undefined
\documentclass[beamer]{IEEEtran}
\usepackage{circuitikz}
\usepackage{cite}
\usepackage{amsmath,amssymb,amsfonts,amsthm}
\usepackage{algorithmic}
\usepackage{graphicx}
\usepackage{textcomp}
\usepackage{xcolor}
\usepackage{txfonts}
\usepackage{listings}
\usepackage{enumitem}
\usepackage{mathtools}
\usepackage{gensymb}
\usepackage{comment}
\usepackage[breaklinks=true]{hyperref}
\usepackage{tkz-euclide} 
\usepackage{listings}
\usepackage{gvv}                                        
\def\inputGnumericTable{}                                 
\usepackage[latin1]{inputenc}                                
\usepackage{color}                                            
\usepackage{array}                                            
\usepackage{longtable}                                       
\usepackage{calc}                                             
\usepackage{multirow}                                         
\usepackage{hhline}                                           
\usepackage{ifthen}                                           
\usepackage{lscape}
\usepackage[export]{adjustbox}

\newtheorem{theorem}{Theorem}[section]
\newtheorem{problem}{Problem}
\newtheorem{proposition}{Proposition}[section]
\newtheorem{lemma}{Lemma}[section]
\newtheorem{corollary}[theorem]{Corollary}
\newtheorem{example}{Example}[section]
\newtheorem{definition}[problem]{Definition}
\newcommand{\BEQA}{\begin{eqnarray}}
\newcommand{\EEQA}{\end{eqnarray}}
\newcommand{\define}{\stackrel{\triangle}{=}}
\theoremstyle{remark}
\newtheorem{rem}{Remark}
\begin{document}
\parindent 0px
\bibliographystyle{IEEEtran}

\title{Assignment\\[1ex]12.7 - 8}
\author{EE23BTECH11220 - R.V.S.S Varun$^{}$% <-this % stops a space
}
\maketitle
\newpage
\bigskip

\renewcommand{\thefigure}{\theenumi}
\renewcommand{\thetable}{\theenumi}
\section*{Question}
A charged 30 $\mu$F capacitor is connected to a 27 mH inductor. Suppose the initial charge on the capacitor is 6mC.What is the total energy stored in the circuit initially? What is the
total energy at later time?
\section*{Solution}
\textbf{Given,}


     Initial charge on capacitor is 6mC.
     \begin{figure}[h]


  
    

    \begin{circuitikz}

 % Inductor
  \draw (0,0) to[L, l=$L$] (2,0);
  
  % Capacitor
  \draw (2,0) -- (2,-2) to[C, l=$C$] (0,-2) -- (0,0);



\end{circuitikz}




  
  
       
  
    \caption{Circuit diagram}
     
   \label{fig:12.7.8.1}
\end{figure}
\begin{table}[h]
  \centering
  \begin{tabular}{|c|c|c|}
    \hline
    Symbol & Description & Value\\
    \hline
    q\brak{0^{+}} & Initial charge on capacitor & 6\ mC \\
    \hline
    q&Charge on capacitor&-\\
    \hline
    L & Value of inductance & 27\ mH \\
    \hline
    C & Value of capacitance & $30\ \mu F$ \\
    \hline
    E&Total energy stored in circuit&-\\
    \hline
    $E_L$&Energy stored in inductor&-\\
    \hline
    $E_C$&Energy stored in capacitor&-\\
    \hline
    i&current in the inductor&-\\
    \hline
    I\brak{s}&Laplace transform of i\brak{t}&-\\
    \hline
  \end{tabular}
  \vspace{5pt}
  \caption{Table of parameters}
  \label{tab:12.7.8.1}
\end{table}
\begin{align}
    L\frac{di\brak{t}}{dt}+\frac{1}{C}\smallint_{-\infty}^{t}i\brak{t}dt=0
\end{align}
\begin{align}
      L\frac{di\brak{t}}{dt}+\frac{1}{C}\smallint_{-\infty}^{0}i\brak{t}dt+\frac{1}{C}\smallint_{0}^{t}i\brak{t}dt=0
\end{align}
\vspace{10pt}
Laplace transform with variable s,
\begin{align}
    L{sI\brak{s}}+\frac{1}{C}\frac{q\brak{0^{+}}}{s}+\frac{1}{C}\frac{I\brak{s}}{s}=0
\end{align}
\begin{align}
    I\brak{s}= \frac{-q\brak{0^{+}}}{LCs^2+1}
\end{align}
\vspace{10pt}
To find initial value of current ,
\begin{align}
    i\brak{0^{+}}=\lim_{s\to\infty}[sI\brak{s}]
\end{align}
\begin{align}
    i\brak{0^{+}}=\lim_{s\to\infty}\left[s\frac{-q\brak{0^{+}}}{LCs^2+1}\right]=0
\end{align}
\vspace{10pt}
To find final value of current ,
\begin{align}
     i\brak{\infty}=\lim_{s\to0}[sI\brak{s}]
\end{align}
\begin{align}
   i\brak{\infty}=\lim_{s\to0}\left[s\frac{-q\brak{0^{+}}}{LCs^2+1}\right]=0 
\end{align}
\begin{align}
    i\brak{0^{+}}=i\brak{\infty}=0
\end{align}
Hence,
\begin{align}
    q\brak{0^{+}}=q\brak{\infty}=6\ mC
\end{align}
\begin{align}
    E=E_L+E_C
\end{align}
from \brak{9},
\begin{align}
    E_L=0
\end{align}
\begin{align}
    E_C=\frac{q^2}{2C}
\end{align}
\begin{align}
    E_C=0.6\ J
\end{align}
\begin{align}
    E=0.6\ J
\end{align}
Hence , the total energy stored in the circuit initially and at a later time is 0.6\ J.
\end{document}
